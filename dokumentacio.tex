\documentclass{article}
\usepackage[utf8]{inputenc}
\usepackage{amsmath}
\usepackage{amsfonts}
\usepackage{bead}

\def\div{\;\mathrm{div}\;}

\begin{document}

\fejlec {
    \bead{1}
    \csoport{3}
    \datum{2010. október 5--6.}
    \eha{KABRABI.ELTE}
    \email{kallo.bernat@gmail.com}
    \gyakvez{Szabóné Nacsa Rozália}
    \feladat{3}
    \nev{Kalló Bernát}
    \tanar{Gregorics Tibor}
    \targy{Objektumelvű alkalmazások fejlesztése}
}

\feladatleiras {
    Valósítsuk meg a nagyon nagyszámok típusát! Ábrázoljuk a számokat számjegyeik
    sorozatával, amelyet egy dinamikus helyfoglalású tömbben helyezzünk el, és
    implementáljuk a hatékony összeadás és a szorzás műveleteit! Tegye lehetővé két
    nagyszám típusú változó közötti értékadást!  
}

\tipus(Bignum típus) {
    \tipusertekhalmaz {
        A nagy számokat a tízes számrendszerbeli számjegyeik sorozatával tárolunk:

            \[ Bignum = \{b \in [0..9]^n | b_n \ne 0\} \]

        Az egszerűség kedvéért úgy vesszük, hogy $i>n$ esetén $a_i=0$.
    }
    \tipusmuveletek {
        \muvelet{Összeadás} {
            Két nagyszám összegét az írásbeli összeadáshoz hasonló módon végezzük.

                \[A=Bignum^3 \;(a,b,c) \]
                \[Q=(a=a' \wedge b=b') \]
                \[R=(Q \wedge c=+(a,b))\]
                \[+(a,b)= \left< (a_0 + b_0) \bmod 10, (a_1 + b_1 + (a_{0} + b_{0}) \div
                10), \dots, (a_n + b_n + (a_{n-1} + b_{n-1}) \div 10)\right> \]

        }
        \muvelet{Egész számmal való szorzás} {
            Egy nagyszámot egy (rendes) egész számmal összeszorozhatunk számjegyenként, az
            írásbeli szorzáshoz hasonlóan:

                \[A=Bignum \times \mathbb{N} \times Bignum \; (a,k,b)\]
                \[Q=(a=a' \wedge k=k')\]
                \[R=(Q \wedge b=*(a,k))\]
                \[*(a,k) = \left<(a_0 \cdot k) \bmod 10, (a_1 \cdot k + (a_0 \cdot k) \div
                10) \bmod 10, \dots,\right.\]
                \[\left.(a_n \cdot k + (a_{n-1} \cdot k) \div 10) \bmod 10,
                (a_{n} \cdot k) \div 10\right >\]
        }
        \muvelet{10-hatvánnyal való szorzás} {
        \muvelet{Szorzás} {
            A szorzást visszavezetjük az összeadásra


        }
    }
    \reprezentacio {

    }
    \implementacio {
        \muvelet{Összeadás} {



        }
    }


}


\end{document}















# vim: lbr:tw=90:nocin:si

