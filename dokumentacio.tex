\documentclass{article}
\usepackage[utf8]{inputenc}
\usepackage{amsmath}
\usepackage{amsfonts}
\usepackage{bead}
\usepackage{stuki}

\def\div{\;\mathrm{div}\;}
\DeclareMathOperator{\new}{new}

\begin{document}

\fejlec {
    \bead{1}
    \csoport{3}
    \datum{2010. október 5--6.}
    \eha{KABRABI.ELTE}
    \email{kallo.bernat@gmail.com}
    \gyakvez{Szabóné Nacsa Rozália}
    \feladat{3}
    \nev{Kalló Bernát}
    \tanar{Gregorics Tibor}
    \targy{Objektumelvű alkalmazások fejlesztése}
}

\feladatleiras {
    Valósítsuk meg a nagyon nagyszámok típusát! Ábrázoljuk a számokat számjegyeik
    sorozatával, amelyet egy dinamikus helyfoglalású tömbben helyezzünk el, és
    implementáljuk a hatékony összeadás és a szorzás műveleteit! Tegye lehetővé két
    nagyszám típusú változó közötti értékadást!  
}

\tipus(Bignum típus) {
    \tipusertekhalmaz {
        A természetes számok.
    }
    \tipusmuveletek {
        \muvelet{Összeadás} {
            $e:=a+b$
        }
        \muvelet{Szorzás} {
            $e:=a*b$
        }
        \muvelet{Létrehozás sztringből} {
            A tízes számrendszerbeli alakja alapján hozza létre a számot.
        }
        \muvelet{Sztringgé alakítás} {
            Állítsa elő a tízes számrendszerbeli alakját.
        }
        \muvelet{Beolvasás} {
            Beolvassa a szám tízes számrendszerbeli alakját.
        }
        \muvelet{Kiírás} {
            Írja ki a tízes számrendszerbeli alakját.
        }
    }
    \reprezentacio {
        Egy tömbben ($v$) tároljuk a szám tízes számrendszerbeli számjegyeit. A 0-s index
        az egyes helyiértéknek felel meg.  Külön mezőben ($n$) tároljuk a számjegyek
        számát. Az $n$-edik helyen nem állhat 0, ha a nullát akarjuk reprezentálni, $n=0$
        lesz.

        Például az 1234 reprezentációja $\left(n:4, v:\left<4,3,2,1\right>\right)$. A 0
        reprezentációja $\left(n:0, v:\left<\right>\right)$.

    }
    \implementacio {
        Itt a specifikációban szereplő függvényekhez hozzáveszünk még néhányat, amelyeket
        az összeadáshoz és a szorzáshoz felhasználunk.


        \muvelet{Egy jegy lekérdezése} {
            Visszaadja a szám egy jegyét, vagy 0-t.

            \algoritmus[5cm]{$k:=a.jegy(i)$} {
                \begin{IF}[50]{1}{\stm{i < a.n}}
                    \stm{k:=a.v[i]}
                \ELSE
                    \stm{k:=0}
                \end{IF}
            }
        }
        \muvelet{Összeadás} {
            A nagyszámokat az írásbeli összeadásnak megfelelően adjuk össze a
            legkisebb helyiértéktől kezdve a legnagyobbig. Az eredmény leföljebb eggyel
            lesz hosszabb, mint a hosszabbik operandus, az egyszerűség kedvéért ennyit le
            is foglalunk a tömbnek.

            \algoritmus[6.5cm]{$e:=a+b$} {
                \stm{maxn:=\max(a.n,b.n)}
                \stm{\new e.v[maxn+1]}
                \stm{k:=a.jegy(0) + b.jegy(0)}
                \stm{i:=0}
                \begin{WHILE}{3}{\stm{i\le maxn \vee k>0}}
                    \stm{e.v[i]:=k \bmod 10}
                    \stm{i:=i+1}
                    \stm{k:=k \div 10 +a.jegy(i) + b.jegy(i)}
                \end{WHILE}
                \stm{e.n:=i}
            }
        }
        \muvelet{Egész számmal szorzás} {
            A nagyszámokat (rendes) egész számmal az írásbeli szorzásnak megfelelően
            szorzunk, számjegyenként a legkisebb helyiértéktől kezdve, nagyon hasonlóan az
            összeadáshoz. Itt is megtehetjük, hogy rögtön a szám hosszánál eggyel nagyobb
            tömböt foglalunk le, mert ennél hosszabb nem lehet a szorzat. Viszont a szám
            hosszánál lehet rövidebb, akkor, ha a szorzó 0, ezért ezt külön meg kell
            nézni.

            \algoritmus[6.5cm]{$e:=a*x$} {
                \begin{IF}[20]{8}{\stm{x=0}}
                    \stm{e.n=0}
                \ELSE
                    \stm{\new e.v[a.n+1]}
                    \stm{k:=a.jegy(0)*x}
                    \stm{i:=0}
                    \begin{WHILE}{3}{\stm{i\le a.n \vee k>0}}
                        \stm{e.v[i]:=k \bmod 10}
                        \stm{i:=i+1}
                        \stm{k:=k \div 10 + a.jegy(i)*x}
                    \end{WHILE}
                    \stm{e.n:=i}
                \end{IF}
            }
        }
        \muvelet{Tízhatvánnyal szorzás} {
            Egy nagyszámot tíz hatványával szorozni hatékonyabban is lehet, mint az előző
            algoitmussal:

            \algoritmus[4cm]{$e:=a*10^x$} {
                \stm{\new e.v[a.n+x]}
                \begin{WHILE}{1}{\stm{i:=0..x-1}}
                    \stm{e.v[i]:=0}
                \end{WHILE}
                \begin{WHILE}{1}{\stm{i:=x..a.n+x-1}}
                    \stm{e.v[i]:=a.v[i-x]}
                \end{WHILE}
                \stm{e.n:=a.n+x}
            }
        }

        \muvelet{Szorzás} {
            A szorzást vissza tudjuk vezetni a számjeggyel való szorzásra, a tízhatvánnyal
            való szorzásra és az összeadásra, az írásbeli szorzásnak megfelelően.

            \algoritmus[5cm]{$e:=a*b$} {
                \stm{e.n:=0}
                \begin{WHILE}{3}{\stm{i:=0..b.n-1}}
                    \stm{p:=a*b.jegy(i)}
                    \stm{p:=p*10^i}
                    \stm{e:=e+p}
                \end{WHILE}
            }
        }

        \muvelet{Létrehozás sztringből} {
            \algoritmus[7.6cm]{$e:=Bignum(s)$} {
                \begin{IF}[20]{4}{\stm{s='0'}}
                    \stm{e.n=0}
                \ELSE
                    \stm{e.n:=length(s)}
                    \stm{\new e.v[e.n]}
                    \begin{WHILE}{1}{\stm{i:=0..e.n}}
                        \stm{e.v[i]:=ord(s[n-i-1])-ord('0')}
                    \end{WHILE}
                \end{IF}
            }
            
        }
        \muvelet{Sztringgé alakítás} {
            \algoritmus[7.6cm]{$s:=String(a)$} {
                \begin{IF}[20]{3}{\stm{e.n=0}}
                    \stm{s:='0'}
                \ELSE
                    \stm{setlength(s,a.n)}
                    \begin{WHILE}{1}{\stm{i:=0..a.n}}
                        \stm{s[n-i-1]:=chr(a.v[i])+chr('0')}
                    \end{WHILE}
                \end{IF}
            }
        }
        \muvelet{Beolvasás} {
            \algoritmus[3cm]{$read(a)$} {
                \stm{readLine(s)}
                \stm{a:=Bignum(s)}
            }
        }
        \muvelet{Kiírás} {
            \algoritmus[3cm]{$print(a)$} {
                \stm{print(String(a))}
            }
        }

    }


}

\megvalositas {
    A \texttt{Bignum} osztályt a \texttt{bignum.cpp}/\texttt{bignum.h} fájlokban
    valósítjuk meg. A műveletek nagy részét operátor-felüldefiniálással implementáljuk. A
    további dokumentációt lásd a forráskódban.

    A kódban megvalósítjuk az említetteken kívül az értékadást, a copy konstruktort,
    ill. a destruktort.
}

\teszteles {
    A teszteléshez a CxxTest keretrendszert használjuk. Ez a \texttt{cxxtest/} mappában található.
    A tesztek a \texttt{test/test.h} fájlban vannak, ebből generáltuk a
    \texttt{test/runner.cpp} fájlt, amit lefordítva kaptuk meg a \texttt{test/runner}
    futtatható állományt. A generáláshoz és fordításhoz szükséges parancsok a Makefile-ban
    láthatók.

    \fekete {
        \item Számok összeadása átvitel nélkül
        \item Számok összeadása átvitellel
        \item Nulla összeadása más számmal, nullával
        \item Számok szorzása
        \item Számok szorzása 1-gyel, 0-val
        \item Nagyon nagy számok ($>2^{32}, 2^{64}$) összeadása, szorzása
    }
    Ezek egyben a sztring és Bignum közötti konverziókat is tesztelik.

    \feher {
        \item Szorzás kommutativitása az előző esetekben (mivel az algoritmusban nem
        szimmetrikusan kezeltük $a$-t és $b$-t)
        \item Szorzáshoz használt segéd metódusok tesztelése
        \item Copy konstruktor tesztelése
        \item Értékadás tesztelése
        \item Destruktor tesztelése
        \item Beolvasás és kiírás tesztelése
    }
}


\end{document}















% vim: lbr:tw=90:nocin:si

